%%%%%%%%%%%%%%%%%%%%%%%%%%%%%%%%%%%%%%%%%
% Medium Length Professional CV
% LaTeX Template
% Version 2.0 (8/5/13)
%
% This template has been downloaded from:
% http://www.LaTeXTemplates.com
%
% Original author:
% Trey Hunner (http://www.treyhunner.com/)
%
% Important note:
% This template requires the resume.cls file to be in the same directory as the
% .tex file. The resume.cls file provides the resume style used for structuring the
% document.
%
%%%%%%%%%%%%%%%%%%%%%%%%%%%%%%%%%%%%%%%%%

%----------------------------------------------------------------------------------------
%	PACKAGES AND OTHER DOCUMENT CONFIGURATIONS
%----------------------------------------------------------------------------------------

\documentclass{resume} % Use the custom resume.cls style

% symbols like \Telefon, \Mobilefone, \Letter and \Email
\RequirePackage{marvosym}
\usepackage[left=0.75in,top=0.6in,right=0.75in,bottom=0.6in]{geometry} % Document margins
\usepackage[spanish]{babel}
\usepackage[utf8]{inputenc}
\usepackage[T1]{fontenc}

\name{Ricardo M. Vilchis} % Your name
%\address{123 Broadway \\ City, State 12345} % Your address
\address{\Letter \   {\bf r}@rvilchis.com \\ \ \Mobilefone \ (+52) 55-4217-0422 } % Your phone number and email

\begin{document}

%----------------------------------------------------------------------------------------
%  PERSONAL DATA
%----------------------------------------------------------------------------------------

\begin{rSection}{Datos personales.}

    Idiomas: Inglés [Intermedio | Reading, listening ]

    \begin{tabular}{l c r}
        \parbox{0.5\textwidth}{
            \begin{itemize}
                \item Domicilio: Miguel Hidalgo, México, D.F.
            \end{itemize}
        } &
        \parbox{0.4\textwidth}{
            \begin{itemize}
                \item Edad: 27 - Soltero
                \item Pasaporte: Sí
            \end{itemize}
        }
    \end{tabular}
    \\
\end{rSection}
%------------------------------------------------

%----------------------------------------------------------------------------------------
%  Specialitys
%----------------------------------------------------------------------------------------

\begin{rSection}{Habilidades especializadas.}

    \begin{tabular}{l c r}
        \parbox{0.5\textwidth}{
            \begin{itemize}
                \item Python
                \item Javascript
                \item HTML, XML
                \item CSS
                \item MySQL
                \item PostgreSQL
                \item Apache HTTP Server
                \item Nginx
            \end{itemize}
        } &
        \parbox{0.4\textwidth}{
            \begin{itemize}
                \item Plone
                \item Django
                \item Pyramid
                \item Backbone.js
                \item RequireJS
                \item Linux (Debian, RHEL)
                \item Git, Mercurial
            \end{itemize}
        }
    \end{tabular}
    \\
\end{rSection}
%------------------------------------------------

%----------------------------------------------------------------------------------------
%	WORK EXPERIENCE SECTION
%----------------------------------------------------------------------------------------
\begin{rSection}{Experiencia}

\begin{rSubsection}{UTEL University. }{Marzo 2014 - Actual}{Web Developer}{México, D.F.}
    \item Implementación de sincronización de sistema de control escolar con aula virtual (Django-Moodle)
    \item Implementación de sistema de pagos con PayU.
    \item Migración y adaptación de datos de sistema escolar al sistema de admisiones. Consistió en la unión de dos bases de datos (MySQL) y dos aplicaciones Django en una sola.
    \item Implementé sistema de autenticación por terceros con OAuth2 (Facebook, Google y YAHOO).
\end{rSubsection}
%------------------------------------------------
\begin{rSubsection}{Alfaomega Grupo Editor. }{Agosto  2008 - Marzo 2014}{Web Developer}{México, D.F.}
    \item Desarrollé, diseñé e implementé aplicación {\em Balanced Score Card} para evaluación de personal. \\
        Tecnologías usadas: Django (south, xadmin, Django REST framework), PostgreSQL, Javascript ( RequireJS, Backbone, JQuery ).
    \item Sugerí y supervicé el desarrollo de una interfaz de captura de datos más accesible para CRM (Vtiger). Usando REST Services y Backbone.js.
    \item Administré y formé parte del equipo de desarrollo del sitio para la publicación de material de apoyo y libros digitales. Sitio basado en Plone.
    \item Desarrollé un sistema de registro de capacitación magisterial, diseñado para la DGENAM. Hecho con Django y MySQL.
\end{rSubsection}

%------------------------------------------------

% \begin{rSubsection}{}{Julio 2008 - Julio 2012}{Web Developer}{}
% \item Diseñé e implementé sistema para crear contenido independiente del aspecto visual usando la especificación DITA (xml), XSLT, HTML, CSS, Javascript.
% \item Desarrollé sistema para crear ejercicios de tipo: Selección simple, selección múltiple, arrastrar y soltar, relación de columnas. Con integración en paquetes SCORM 1.2.
% \item Desarrollé objetos virtuales de aprendizaje y simuladores como apoyo al contenido de diferentes libros: métodos de ordenamiento, Máquina de turing, grafos,  álgebra booleana,  árboles,
        % complejidad computacional, conjuntos, algoritmo de fleury,  máquina de estado finito,  motor de inducción polifásico, sistemas numéricos. Todos desarrollados usando ActionScript3
% \end{rSubsection}
\end{rSection}

%----------------------------------------------------------------------------------------
%	TECHNICAL STRENGTHS SECTION
%----------------------------------------------------------------------------------------

%\begin{rSection}{Technical Strengths}
%
%\begin{tabular}{ @{} >{\bfseries}l @{\hspace{6ex}} l }
%Computer Languages & Prolog, Haskell, AWK, Erlang, Scheme, ML \\
%Protocols \& APIs & XML, JSON, SOAP, REST \\
%Databases & MySQL, PostgreSQL, Microsoft SQL \\
%Tools & SVN, Vim, Emacs
%\end{tabular}
%
%\end{rSection}


%----------------------------------------------------------------------------------------
%	EDUCATION SECTION
%----------------------------------------------------------------------------------------

\begin{rSection}{Educación}

{\bf Escuela Superior de Cómputo. IPN} \hfill {\em Junio 2012} \\
Ingeniería en Sistemas Computacionales \\

\end{rSection}

%----------------------------------------------------------------------------------------
%	EXAMPLE SECTION
%----------------------------------------------------------------------------------------

%\begin{rSection}{Section Name}

%Section content\ldots

%\end{rSection}

%----------------------------------------------------------------------------------------

\end{document}
